\documentclass[11pt]{article}

\setlength\topmargin{-0.6cm}   
\setlength\textheight{23.4cm}
\setlength\textwidth{17.0cm}
\setlength\oddsidemargin{0cm} 
 

\begin{document}

\begin{center}
\LARGE
LING572 Hw6: Beam search\\
Due: 11pm on Feb 20, 2020\\
\vspace{0.3in}
\end{center}


The example files are under /dropbox/19-20/572/hw6/examples/.


%%%%%%%%%%%%%%%%%%%%%%%%
\vspace{0.4 in}

\noindent {\bf Q1 (75 points):} 
  Write a script, {\bf beamsearch\_maxent.sh}, that implements
  the beam search for POS tagging.

\begin{itemize}
  \item The format is: beamsearch\_maxent.sh test\_data boundary\_file model\_file sys\_output beam\_size topN topK

  \item test\_data has the following format (e.g., {\bf ex/test.txt}):
        ``instanceName goldClass f1 v1 f2 v2 ...'', where 
          an instance corresponds to a word and goldClass is
          the word's POS tag according to the gold standard. 
          Note this format is slightly different
         from the format used in the previous assignments, which is 
         ``goldClass f1:v1 f2:v2 ...''.

  \item boundary\_file: the format of boundary\_file is 
       one number per line, which is the length of a sentence
       (e.g., {\bf ex/boundary.txt}); for instance, if the first line
       is 46, it means the first sentence in test\_data has 46 words.
       
  \item model\_file is a MaxEnt model in text format (e.g., {\bf m1.txt}). 
 
  \item sys\_output (e.g., {\bf ex/sys}) has the following format: 
         ``instanceName goldClass sysClass prob'', where 
         $instanceName$ and $goldClass$ are copied from the test\_data, 
         $sysClass$ is the tag $y$ for the word $x$ according to the best 
         tag sequence found by the beam search, 
         and $prob$ is $P(y \mid x)$. Note $prob$ is NOT the probability 
         of the whole tag sequence given the word sentence. It is the
         probability of the tag $y$ given the word $x$.

  \item topN: When expanding a node 
        in the beam search tree, choose only the 
        $topN$ POS tags for the given word based on 
        $P(y \mid x)$.

  \item beam\_size is the max gap between the lg-prob of the best path
        and the lg-prob of kept path: 
        that is, a kept path should satisfy
        $lg(prob) + beam\_size \ge lg(max\_prob)$, where $max\_prob$ is 
        the prob of the best path for the current position.
        $lg$ is base-10 log.
        
  \item topK is the max number of paths kept alive
         at each position after pruning. 

\end{itemize}


Note:
 \begin{itemize}
   \item A {\it path} in the beam search is the path from the root
         to a node in the beam search tree.
         And for more info about how beam search works and the meaning of 
         beam\_size, topN and topK, see the hw6 slides.

   
   \item Remember that the feature vectors in the test\_data do not 
          include features $t_{i-1}$=$tag_{i-1}$ (e.g., {\bf prevT=NN})
          and $t_{i-2} \ t_{i-1}$=$tag_{i-2}+tag_{i-1}$ 
          (e.g., {\bf prevTwoTags=JJ+NN}), because the tags of the previous
          words are not available for the test data before the decoding 
          starts. You need to 
          add those features to the feature vectors before 
          calling the model to classify the current instance based on 
          the current path. 

     \begin{itemize}
      \item   For instance, suppose the current instance is ``instanceName goldTag
          f1 v1 f2 v2 ...'', and in the current path
          the system tags the previous word as NN
          and the word before the previous word as JJ. You need to 
          add ``prevT=NN 1'' and ``prevTwoTags=JJ+NN 1'' to the feature vector
          in order to determine the top tags of the current instance
          according to the current path.
          
       \item When you add these two types of features, only add the ones 
        that appear in the model file. If a feature (e.g., prevTwoTags=NN+RB)
        does not appear in the model file, that means that the tag bigram
        does not appear in the training data. In that case, do not add
        the feature to the feature vector, as the model does not contain
        the weights for the corresponding feature functions. Another way
        to look at this is that if a (feature, class) pair does not appear
        in the model file, it means the weight of the feature function
        is zero.

       \item For your convenience, the list of these two types of features 
        in the {\bf m1.txt} is stored in {\bf feats\_to\_add}. 
        Your code should NOT read in a file like {\bf feats\_to\_add}
        because this info should come from the model file. 
        This file is there just to show you what these features look like.

      \item To summarize, you need to add prevT=xx and
        prevTwoTags=yy+xx features on the fly.
        If such a feature does not appear in the model
        file,  simply ignore the feature (i.e., assuming its weight is 0).
    \end{itemize}

\end{itemize}

        
\noindent
Run beamsearch\_maxent.sh with 
      {\bf sec19\_21.txt} as the test data,  {\bf m1.txt} as model\_file,
      {\bf sec19\_21.boundary} as the boundary file. 
\begin{itemize}
\item Before running your code on the whole test set,
  you should test your code on smaller data sets.
  For instance, you can use ex/test.txt as the test file,
  ex/boundary.txt as boundary file,
  m1.txt as the model file.
  After that, you can run your code on the real data set with the (0, 1, 1)
  setting, and record the time it takes.
  The running time for other settings could be much longer.

    \item Fill out Table 1.
    \item Submit the sys\_output file for the third row in Table 1
          (i.e., the row when beam\_size=2, topN=5, and topK=10).
\end{itemize}


\begin{table}[h]
\centering
\begin{tabular}{|r|l|l|l|l|} \hline
beam\_size & topN  & topK  & Test accuracy  & Running time \\  \hline
0          & 1     & 1     &    & \\ \hline
1          & 3     & 5     &    & \\ \hline
2          & 5     & 10    &    &  \\ \hline
3          & 10    & 100   &    &  \\ \hline 
\end{tabular}
\caption{Beam search results}
\label{table1}
\end{table}


 

%%%%%%%%%%%%%%%%%%
\vspace{0.5 in}

\noindent
{\bf Submission:}  Submit the following to Canvas:
\begin{itemize}
  \item Your note file {\it readme.(txt $\mid$ pdf)}
        that includes Table 1 and any notes that you want the TA to read.
      

  \item  hw.tar.gz that includes all the files specified in
      dropbox/19-20/572/hw6/submit-file-list, plus any source code
      (and binary code) used by the shell scripts.

  \item Make sure that you run {\bf check\_hw6.sh} before
    submitting your hw.tar.gz.


\end{itemize}

\end{document}




\documentclass[11pt]{article}

% \setlength\topmargin{-0.6cm}   
% \setlength\textheight{23.4cm}
% \setlength\textwidth{17.0cm}
% \setlength\oddsidemargin{0cm} 

\usepackage[margin=1in]{geometry}
\usepackage{hyperref}
 

\begin{document}

\begin{center}
\LARGE
LING 572 HW2\\
Due: 11pm on Jan 23, 2020\\
\end{center}


The example files are under ~/dropbox/19-20/572/hw2/examples/.

\vspace{0.3 in}
\noindent {\bf Q1 (4 points):} Run the Mallet DT learner (i.e., the trainer's name
is DecisionTree) with 
{\bf train.vectors.txt} as the training data 
and {\bf test.vectors.txt} as the test data.
%
In your note file, write down the following:
\begin{description}
\item [(a)] The command lines you use for preparing data, training, testing, 
            and getting the training and test accuracy.
            You can use vectors2classify commands
            to do training, testing and evaluation in one step.

\item [(b)] What are the training accuracy and the test accuracy?
\end{description}



%%%%%%%%
\vspace{0.7 in}

\noindent {\bf Q2 (6 points):} Run the Mallet DT trainer with different depths; that is,
when running vectors2classify, replace 
{\tt --trainer DecisionTree}
with
\begin{verbatim}
  --trainer "new DecisionTreeTrainer(nn)"
\end{verbatim}
where nn is the depth of the decision tree. Note that you have to 
use vectors2classify, instead of ``mallet train-classifier'' and
``mallet classify-svmlight'' because ``mallet train-classifier''
does not process "new DecisionTreeTrainer(nn)" properly.


\begin{description}
\item [(a)] Fill out Table 1
\item [(b)] What conclusion can you draw from Table 1?
\end{description}

\begin{table}[h]
\centering
\caption{Run Mallet's DT learner with different depths}
\label{table1}
\begin{tabular}{|l|l|l|} \hline
Depth   & Training accuracy & Test accuracy \\ \hline
1       &    &  \\ \hline
2       &    &  \\ \hline
4       &    &  \\ \hline
10      &    &  \\ \hline
20      &    &  \\ \hline
50      &    &  \\ \hline
\end{tabular}

\end{table}




%%%%%%%%%%%%%%%%%%%%%%%%%%%%%%%%%%%%%%%%%%%%
\vspace{0.7 in}

\noindent {\bf Q3 (55 points):} Write a program, {\bf build\_dt.sh}, 
that builds a DT tree from
the training data, classifies the training and test data, and calculates
the accuracy.
\begin{itemize} 
   \item This DT learner should treat all features as binary; that is,
        the feature is considered present if its value is nonzero, and
        absent if its value is zero.

   \item Use information gain to select features when building the DT.

  \item The format of the command line would be:

	  {\tt build\_dt.sh training\_data test\_data max\_depth min\_gain model\_file sys\_output $>$ acc\_file }

  \item training\_data and test\_data are the vector files in the text format
        (cf. {\bf train.vectors.txt}).

  \item max\_depth is the maximum depth of the DT,\footnote{The depth of 
        the root is 0, the depth of its children is 1, and so on.}
        and min\_gain is the minimal gain. Those parameters are used
        to determine when to stop building the DT; that is, 
        split the current training data set at the node x
        if and only if (the depth of x $<$ max\_depth)
        AND (the infoGain of the split $\ge$ min\_gain).

      \item model\_file is the DT tree (cf. {\bf model\_ex}) produced by
        the DT trainer.
        Each line corresponds to a leaf node in the DT and it has the format: 
        path training\_instance\_num c1 p1 c2 p2 ...\\
        Where path is the path from the root to the leaf node, 
        training\_instance\_num is the number of the training examples
        that ``reach'' the leaf node, $c_i$ is the class label, and
        $p_i$ is the probability of $c_i$ (i.e., the percentage of 
        the training examples at the leaf node with the label $c_i$).

  \item sys\_output is the classification result on the training and
        test data (cf. {\bf sys\_ex}).
        Each line has the following format: \\
	{\tt instanceName c1 p1 c2 p2 ...} \\
        where instanceName is just something
        like ``array:0'', ``array:1''.
        
  \item acc\_file shows the confusion matrix and the accuracy for
       the training and the test data (cf. {\bf acc\_ex}).
       In the confusion matrix, a[i][j] is the number of instances
        where the truth is class i, and the system output is class j.

   \item As always, model\_ex, sys\_ex, and acc\_ex in the examples/ directory
         are NOT gold standard.
        These files were created just to show you the format of the files.
\end{itemize}

\noindent Run {\bf build\_dt.sh} 
with {\bf train.vectors.txt} as the 
training data and {\bf test.vectors.txt} as the test data:
\begin{itemize}
  \item Fill out Table 2 (where min\_gain is set to 0) and Table 3 
    (where min\_gain is set to 0.1).
  \item submit model\_file, sys\_output, acc\_file produced by running \\
    {\tt build\_dt.sh train.vectors.txt test.vectors.txt 4 0.1 model\_file sys\_output $>$ acc\_file}
\end{itemize}


\begin{table}[hbtp]
\centering
\caption{Your decision tree results when min\_gain=0}
\label{table2}
\begin{tabular}{|l|l|l|l|} \hline
Depth   & Training accuracy & Test accuracy & CPU time (in minutes)\\ \hline
1       &  &  &  \\ \hline
2       &  &  &  \\ \hline
4       &  &  &  \\ \hline
10      &  &  &  \\ \hline
20      &  &  &  \\ \hline
50      &  &  &       \\ \hline
\end{tabular}
\end{table}


\begin{table}[hbtp]
\centering
\caption{Your decision tree results when min\_gain=0.1}
\label{table3}
\begin{tabular}{|l|l|l|l|} \hline
Depth   & Training accuracy & Test accuracy & CPU time (in minutes)\\ \hline
1       &  &  &  \\ \hline
2       &  &  &  \\ \hline
4       &  &  &  \\ \hline
10      &  &  &  \\ \hline
20      &  &  &  \\ \hline
50      &  &  &       \\ \hline
\end{tabular}
\end{table}


%%%%%%%%%%%%%%%%%%%%%%%%
\vspace{0.7 in}

\noindent {\bf Q4 (5 points):} Slide \#12 of class2\_DT.pdf 
shows a DT: f1 and f2 are two features; 
f1 is in [-20, 30]; f2 is in [-10, 30].
$L_i$ (i=1, ..., 7) represents a leaf node.
Each leaf node corresponds to a rectangle in 
a 2-dimensional space, where f1 is the x-axis and f2
is the y-axis. Draw a graph that shows the boundary 
of the seven rectangles in this 2-dimensional space.


%%%%%%%%%%%%%%%%%%%%%%%%%%%%%%%%%%%
\vspace{0.7 in}
\noindent
{\bf Q5 (5 ``free'' points):} 
    If you are not familiar with Patas or Condor submit, please 
    go over the condor tutorial at
    \href{https://www.shane.st/teaching/571/aut19/welcome_to_patas_1920.pdf}{https://www.shane.st/teaching/571/aut19/welcome\_to\_patas\_1920.pdf}
    (linked under ``Resources'' on course page).
    You can run condor\_submit for the code in Q3.
    We will use condor\_submit for many assignments later.
    

%%%%%%%%%%%%%%%%%%%%%%%%%%%%%%%%%%%%%%%%%%%%%%%%%%%
\vspace{2in}
\hspace{-0.3in}
{\bf Submission:} Submit the following to Canvas:

\begin{itemize}
  \item Your note file {\it readme.(txt $\mid$ pdf)}
      that includes your answers to Q1-Q4, 
      and any notes that you want the TA to read.

    \item hw2.tar.gz that includes all the files specified in
	    {\tt dropbox/19-20/572/hw2/submit-file-list}, plus any source code
      (and binary code) used by the shell scripts.

    \item Make sure that you run {\bf check\_hw2.sh} before
      submitting your hw.tar.gz.

    \item No need to submit anything for Q5.
      
\end{itemize}

\end{document}




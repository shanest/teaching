\documentclass[11pt]{article}

\setlength\topmargin{-0.6cm}   
\setlength\textheight{23.4cm}
\setlength\textwidth{17.0cm}
\setlength\oddsidemargin{0cm} 
 

\begin{document}

\begin{center}
\LARGE
LING572 HW8: Neural Networks\\
Due: 11pm on March 12, 2020\\
\vspace{0.3in}
\end{center}

A few notes about this assignment:
\begin{itemize}

 \item The answers to the questions should be pretty short. I've left some space for you to fill in the answers. I've also made the \LaTeX file available in case you want to add the answers to the latex file directly. In that case, you need to run pdf2latex, latexmk, or something like that to generate a pdf from the \LaTeX file.

 \item If you prefer to write formulas on paper (instead of typing them with \LaTeX or Word), it's ok. You just need to fill out the rest of the assignment, print out the file, insert formulas by hand, scan the paper, and then submit via Canvas.

 \item Since no programming is required, you only need to submit a single
     file. Please call it {\bf readme.pdf}.

  \item The assignment has two parts:
    \begin{itemize}
    \item Q1-Q3 on derivatives: Recall that college-level calculus is
      a prerequisite of LING572. If you've forgetten how (partial) derivatives work,
      feel free to check any calculus textbook or review the Wikipedia pages
      on those topics. (Just search for derivatives, gradient, partial derivatives,
      etc.)

    \item Q4-Q6: most topics are covered in class, but do consult the readings as well.
    \end{itemize}
    
\end{itemize}




%%%%%%%%%%%%%%%%%%%
\vspace{0.4 in}
\noindent
       {\bf Q1 (10 points):} Let $f'(x)$ denote the derivative of a
         function $f(x)$ w.r.t. the variable $x$.
\begin{description}
   \item [(a) 2 pts:] What does f'(x) intend to measure? \\
  
   \item [(b) 2 pts:] Let $h(x)=f(g(x))$. What is $h'(x)$? \\

   \item [(c) 2 pts:] Let $h(x)=f(x)g(x)$. What is $h'(x)$? \\

   \item [(d) 2 pts:] Let $f(x)=a^x$, where $a>0$. What is $f'(x)$? \\

   \item [(e) 2 pts:] Let $f(x)= x^{10}-2x^8 + \frac{4}{x^2} + 10$.
            What is $f'(x)$? \\
\end{description}


%%%%%%%%%%%%%%%%%%%
\vspace{0.4 in}
\noindent
{\bf Q2 (15 points):} The logistic function is $f(x)=\frac{1}{1+e^{-x}}$.
       The tanh function is $g(x)=\frac{e^x - e^{-x}}{e^x +e^{-x}}$.
    \begin{description}
     \item [(a) 5 pts:] Prove that $f'(x)=f(x)(1-f(x))$. \\ \\ \\ \\
     \item [(b) 5 pts:] Prove that $g'(x)=1 - g^2(x)$. \\  \\ \\ \\  
     \item [(c) 5 pts:] Prove that $g(x) = 2f(2x)-1$  \\  \\ \\ \\
    \end{description}
    


%%%%%%%%%%%%%%%%%%%
\vspace{0.4 in}
\noindent
       {\bf Q3 (15 points):} Let us denote the partial derivative of a
       multi-variate function $f$ w.r.t. one of its variables x by
       $f'_x$ or $\frac{\partial f}{\partial x}$.         
\begin{description}
\item [(a) 2 pts:] What is $f'_x$ trying to measure? \\
\item [(b) 2 pts:] Let $f(x,y)=x^3 + 3x^2y+y^3 + 2x$. What is $f'_x$?
  What is $f'_y$? \\ \\  \\
\item [(c) 2 pts:] Let $z = \sum_{i=1}^n w_i x_i$.
  What is $\frac{\partial z}{\partial w_i}$? \\ \\
  
\item [(d) 4 pts:] Let $f(z)=\frac{1}{1+e^{-z}}$ and $z = \sum_{i=1}^n w_i x_i$. \\
       What is $\frac{\partial f}{\partial z}$? \\ \\ \\ 
       What is $\frac{\partial f}{\partial w_i}$? \\ \\ \\
       Hint: Use the answers that contain $f(z)$.

\item [(e) 5 pts:] Let $E(z)=\frac{1}{2}(t - f(z))^2$,
       $f(z)=\frac{1}{1+e^{-z}}$ and $z = \sum_{i=1}^n w_i x_i$.
  What is $\frac{\partial E}{\partial w_i}$? Hint: the answer should contain $f(z)$. \\ \\
\end{description}


  
%%%%%%%%%%%%%%%%%%%
\vspace{0.4 in}
\noindent
{\bf Q4 (10 points):} The softmax function:
\begin{description}
\item [(a) 5 pts:] In general where in NNs is the softmax function used and why? \\ \\ \\

\item [(b) 5 pts:] If a vector x is [1, 2, 3, -1, -4, 0], what is the value of
  softmax(x)? \\ \\ \\ 
\end{description}
 


%%%%%%%%%%%%%%%%%%%%
\vspace{0.4in}
\noindent
{\bf Q5 (15 points):} Suppose a feedforward neural network (MLP) has
  $m$ layers: the input layer is the 1st layer, the output layer
  is the last layer, and there are $m-2$ hidden layers in between. 
  The number of neurons in the $i^{th}$ layer is $n_i$.
  Each neuron in one layer is connected to
  every neuron in the next layer and there is no other connection.
\begin{description}
\item [(a) 5 pts:] How many connections (i.e., weights) are there in
                   this network? \\ \\

\item [(b) 10 pts:] Let $x$ be a column vector that denotes the values of the
  input layer. Let $M_k$ denote the weight matrix
  between layer $k$ and $k+1$; that is, the cell $a_{i,j}$
  in $M_k$ stores
  the weight on the arc from the $j^{th}$ neuron in layer $k$
  to the $i^{th}$ neuron in layer $k+1$. Let $g$ be the activation function used
  in each layer.
  \begin{itemize}
    \item Given the input $x$, what is the formula for calculating
      the output of the first hidden layer?  \\ \\

    \item Given the input $x$, what is the formula for calculating
      the output of the output layer? \\ \\ \\
        
    \item Hint: In class, we show the formula for calculating the $z$ and $y$
          value for a neuron, where $z = b + \sum_j w_j x_j$ and $y=g(z)$.
          Now there are $n_2$ neurons in the 2nd layer.
          The output of this layer, $y$,
          is going to be a column vector, not a real number. The weights between
          the two layers are no longer a vector,
          but a $n_2 \times n_1$ matrix denoted by $M_1$. So the answer
          to the 1st question should be a simple formula
          that uses matrix operations. For the sake of simplicity, let's assume
          the bias $b$ is always zero.
        \item Terminology: A row vector is a $1 \times n$ matrix
          (e.g., $[a_1, a_2, ..., a_n]$); a column vector is
          a $n \times 1$ matrix. If you transpose a row vector, you get a
          column vector.
   \end{itemize}

\end{description}  



%%%%%%%%%%%%%%%%%%%%%%
\vspace{0.4in}
\noindent
{\bf Q6 (35 points):} Suppose that you were training a neural network to do text classification, with $n > 2$ classes.

\begin{description}
\item [(a) 5 pts:] What loss function would you use? Why would you minimize this function instead of maximizing classification accuracy? \\

\item [(b) 5 pts:] In gradient descent, what's the formula for updating
   the weight matrix (or vector)? And why is that a good formula? \\ \\


\item [(c) 15 pts:] What are the main idea and benefit
  of stochastic gradient descent? \\ \\

  What is a training epoch? \\ \\

  Let $T$ be the size of the training data, $m$ be the size of mini-batch,
  and your training process contains $E$ training epoches.
  How many times is each weight in the NN updated? \\ \\ 


\item [(d) 10 pts:] How can one choose the learning rate? What's the risk
  if the rate is too big? What's the risk if the rate is too
  small? \\ \\ 

\end{description}



           

           





%%%%%%%%%%%%%%%%%%
\vspace{0.5 in}
\noindent
{\bf Submission:}  Submit the following to Canvas:

\begin{itemize}
\item Since HW8 has no coding part, you only need to submit your {\bf readme.pdf}
  which includes answers to all the questions, plus anything you want
  TA to know. No need to submit anything else. 
  
\end{itemize}

\end{document}

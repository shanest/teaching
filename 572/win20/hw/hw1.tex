\documentclass[11pt]{article}
\usepackage{url}
\setlength\topmargin{-0.6cm}   
\setlength\textheight{23.4cm}
\setlength\textwidth{17.0cm}
\setlength\oddsidemargin{0cm} 
 

\begin{document}

\begin{center}
\LARGE
LING 572 HW1\\
Due: 11pm on Jan 16, 2020\\
\vspace{0.3in}
\end{center}


\vspace{0.2 in}
\hspace{-0.3in}
{\bf Q1 (25 points):} Let X and Y be two random variables. 
  The values for P(X,Y) are shown in Table \ref{q1-table1},
  H(X) is the entropy of X, and MI(X,Y) is the mutual information of X and Y.
  Please write down the formulas and the results for the following.

\begin{description}
   \item [(a) 1 pt:] $P(X)$
   \item [(b) 1 pt:] $P(Y)$
   \item [(c) 2 pt:] $P(X \mid Y)$
   \item [(d) 2 pt:] $P(Y \mid X)$
   \item [(e) 2 pts:] Are X and Y independent? Why or why not?

   \item [(f) 2 pts:] $H(X)$
   \item [(g) 2 pts:] $H(Y)$
   \item [(h) 2 pts:] $H(X,Y)$
   \item [(i) 2 pts:] $H(X \mid Y)$
   \item [(j) 2 pts:] $H(Y \mid X)$

   \item [(k) 2 pts:] $MI(X,Y)$
   \item [(l) 5 pts:] 
       The value for $Q(X,Y)$ are shown in Table \ref{q1-table2}.
                What is the value for $KL(P(X,Y) \mid\mid Q(X,Y))$?
                What is the value for $KL(Q(X,Y) \mid\mid P(X,Y))$?
                Are they the same?
\end{description}

\begin{table}[btph]
\centering
\caption{The joint probability P(X,Y)}
\label{q1-table1}
\begin{tabular}{|r|r|r|r|}  \hline
    & X=1 & X=2 & X=3 \\ \hline
Y=a & 0.10 & 0.20 &  0.30 \\ \hline 
Y=b & 0.05 & 0.15 &  0.20 \\ \hline
\end{tabular}
\end{table}


\begin{table}[btph]
\centering
\caption{The joint probability Q(X,Y)}
\label{q1-table2}
\begin{tabular}{|r|r|r|r|}  \hline
    & X=1 & X=2 & X=3 \\ \hline
Y=a & 0.10 & 0.20 &  0.40 \\ \hline 
Y=b & 0.01 & 0.09 &  0.20 \\ \hline
\end{tabular}
\end{table}


%%%%%%%%%%%%%%%%
\vspace{0.3 in}
\hspace{-0.3in}
{\bf Q2 (10 points):} Let $X$ be a random variable for the result of 
  tossing a coin. $P(X=h)=p$; that is, $p$ is the possibility of getting 
  a head, and $1-p$ is the possibility of getting a tail. 
\begin{description}
  \item [(a) 1 pt:] $H(X)$ is the entropy of $X$. Write down the formula for $H(X)$.
  \item [(b) 2 pts:] Let $p^* = arg \ max_{p} \ H(X)$; that is, $p^*$ is the p 
               that results in the maximal value of $H(X)$. What is $p^*$?

  \item [(c) 7 pts:] Prove that the answer you give in (b) is correct.
               Hint: recall how you calculate the optimal solution for 
                   a function $f(x)$ in your calculus class.
                   In this case, $H(X)$ is a function of $p$.
\end{description}



%%%%%%%%%%%%%%%%
\vspace{0.3 in}
\hspace{-0.3in}
{\bf Q3 (25 points):} Permutations and combinations:
   \begin{description}
     \item [(a) 6 pts:] The class has n students, and n is an even number.
        The students are forming teams to work on their homework. Each team
        has exactly 2 students and each student has to appear in exactly one 
        team. How many distinct ways are there to form the teams
        for the class? Write down the formula. Hint: when n=4, there are 3 ways.
        For instance, if student \#1 and \#2 are in the same team, 
        students \#3 and \#4 would have to be in the same team too.


     \item [(b) 5 pts:] There are 10 balls: 
    5 are red, 3 are blue, and 2 are white. 
    Suppose you put the balls in a line, 
    how many different color sequences are there?
    
     \item [(c) 14 pts:] Suppose you want to create a document of 
        length $N$ by using only the words in a vocabulary 
        $\Sigma=\{w_1, w_2, ..., w_n\}$. Let $[t_1, t_2, ..., t_n]$
        be a list of non-negative integers such that $\sum_i t_i = N$.
       \begin{description}
         \item [(c1) 7 points:] 
        How many different documents are there which satisfy
       the condition that, 
       for each $w_i$ in the vocabulary $\Sigma$, the occurrence of the word
       $w_i$ in the document is exactly $t_i$? That is,
      how many different word sequences are there which contain 
      exactly $t_i$ $w_i$'s for each $w_i$ in $\Sigma$? \\

      Hint: The answer to (c1) is very similar to the answer to (b). \\
 
    \item [(c2) 7 pts:] Let P(X) be a unigram model on the vocabulary $\Sigma$;
      that is, $P(X=w_i)$ is the probability of a word $w_i$,
      and $\sum_{w_i \in \Sigma} P(X=w_i)=1$. \\
      
      Suppose a document of length $N$ is created with 
        the following procedure: for each position in a document,
        you pick a word from the vocabulary according to P(X);
        that is, the probability of picking $w_i$ is $P(X=w_i)$. 
        What is the probability that you will end up with a document where
         the occurrence of the word $w_i$ (for each $w_i \in \Sigma$)
         in the document is exactly $t_i$? \\

         Hint: As (c1) shows, there will be many 
         documents that contain exactly $t_i$ $w_i$'s. The answer to (c2) should
         be the sum of the probabilities of all these documents.
      \end{description}
 
  \end{description}

\vspace{0.3 in}
\hspace{-0.3in}
{\bf Q4 (10 points):} Suppose you want to build a trigram POS tagger.
      Let T be the size of the tagset and V be the size of the vocabulary.
  \begin{description}
   \item [(a) 2 pts:] Write down the formula for calculating 
               $P(w_1, ..., w_n, t_1, ..., t_n)$, 
               where $w_i$ is the i-th word in a sentence,
               and $t_i$ is the POS tag for $w_i$.
   %\vspace{1in}

   \item [(b) 8 pts:] Suppose you will use an HMM package to implement
               a trigram POS tagger.
        \begin{itemize}
        \item What does each state in HMM correspond to?
              How many states are there?

         %\vspace{0.5in}
         \item What probabilities in the formula for (a) do 
                transition probability $a_{ij}$
                and emission probability $b_{jk}$ correspond to? 
                $a_{i,j}$ is the transition probability from state 
                $s_i$ to $s_j$, and $b_{jk}$ is the probability that 
                State $s_j$ emits symbol $o_k$.

         %\vspace{0.5in}

        \end{itemize}

   \end{description}


\vspace{0.3in}
\hspace{-0.3in}
{\bf Q5 (10 points):} In a POS tagging task, let V be the
size of the vocabulary (i.e., the number of words), and 
T be the size of the tagset.
Suppose we want to build a classifier that predicts the tag of 
  the current word by using the following features:
\begin{enumerate}
  \item Previous word $w_{-1}$
  \item Current word $w_0$
  \item Next word $w_{+1}$
  \item Surrounding words $w_{-1} \ w_{+1}$
  \item Previous tag $t_{-1}$
  \item Previous two tags $t_{-2} \ t_{-1}$
\end{enumerate}

\begin{description}
\item [(a) 3 pts:] How many unique features are there {\bf in total}?
       You just need to give the answer in the Big-O notation (e.g., $O(V^3)$).
  %\vspace{1.5in}

  \item [(b) 2 pts:] A classifier predicts class label y given the input x.
        In this task, what is x? what is y?

  %\vspace{1.5in}

  \item [(c) 5 pts:] For the sentence {\bf Mike/NN likes/VBP cats/NNS}, 
  write down the feature vector for each word in the sentence.
  The feature vector has the format 
     ``InstanceName classLabel featName1 val1 featName2 val2 ....''.
     For the instanceName, just use the current word.

\end{description} 




 


%%%%%%%%%%%%%%%%%%%%
\vspace{0.3 in}
\hspace{-0.3in}
       {\bf Q6 (10 points):} Suppose you want to build a language identifier (LangID)
       that determines the language code of a given document.
  The training data is a set of documents with the language code for each
  document specified. The test data is a set of documents, and your LangID
  needs to determine the language code of each document. 

\begin{description}
  \item [(a) 7 pts:] How do you plan to build the LangID system? For instance, 
       if you want to treat this as a classification problem, 
       what would x (the input) be? what would y (the output) be?
       What would be good features? Name at least five types of features
       (e.g., one feature type is the word unigrams in the document).

  \item [(b) 3 pts:] 
          What factors (e.g., the amount of training data) could affect
          the system performance? Name at least three factors, excluding
          the amount of training data.
\end{description}



%%%%%%%%%%%%%%%%%%%%
\vspace{0.3 in}
\hspace{-0.3in}
{\bf Q7 (10 ``free'' points):}
    If you are not familiar with Mallet, please go over the Mallet slides
    at the course website\footnote{``Background'' on the first day in the schedule.}
    
    Set up the package in your patas 
    environment, run some experiments. We will use Mallet in later
    assignments.  If you do not have a patas account, you should contact
    me right away.



%%%%%%%%%%%%%%%%%%%%%%%%%%%%%%%
\newpage
\hspace{-0.3in}
{\bf Submission:} In your submission, include the following:
\begin{itemize}
  \item readme.(txt$\mid$pdf) that includes your answers to Q1-Q6. 
    No need to submit anything for Q7.
    
  \item Since this assignment does not require programming,
      there is no need to submit hw.tar.gz,
      and no need to run check\_hwX.sh script.
   
\end{itemize}





\end{document}




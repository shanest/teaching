\documentclass[11pt]{article}

\usepackage[margin=1in]{geometry}
\usepackage{hyperref}


\begin{document}

\begin{center}
\LARGE
LING572 HW3 (Na\"ive Bayes)\\
Due: 11pm on Jan 30, 2020\\
\vspace{0.3in}
\end{center}


The example files are under dropbox/19-20/572/hw3/examples/.

\vspace{0.3 in}
\noindent {\bf Q1 (5 points):} Run the Mallet NB learner (i.e., the trainer's name
is NaiveBayes) with {\bf train.vectors.txt} as the training data 
and {\bf test.vectors.txt} as the test data.
In your note file, write down the training accuracy and the test accuracy.



%%%%%%%%%%%%%%%%%%%%%%%%%%%%%%%%%%%%%%%%%%%%
\vspace{0.4 in}

\noindent {\bf Q2 (35 points):} Write a script, {\bf build\_NB1.sh}, 
that implements the Multi-variate Bernoulli NB model. It builds a
NB model from the training data, classifies the training and test data, 
and calculates the accuracy. 


\begin{itemize} 
  \item The learner should treat all features as binary; that is,
        the feature is considered present iff its value is nonzero.

\item The format is: {\tt build\_NB1.sh training\_data test\_data  class\_prior\_delta cond\_prob\_delta model\_file sys\_output $>$ acc\_file }

  \item training\_data and test\_data are the vector files in the text format
        (cf. {\bf train.vectors.txt}).

  \item class\_prior\_delta is the $\delta$ used in add-$\delta$ smoothing 
         when calculating the class prior $P(c)$; 
         cond\_prob\_delta is the $\delta$ used in add-$\delta$ smoothing 
         when calculating the
         conditional probability $P(f \mid c)$.

  \item model\_file stores the values of 
        P(c) and $P(f \mid c)$ (cf. {\bf model1}). \\
        Comment lines start with ``\%''.
        The line for P(c) has the format 
        ``classname P(c) logprob'', where logprob is 10-based log of P(c). \\
        The line for $P(f \mid c)$ has the format 
        ``featname classname P(f$\mid$c) logprob'', 
         where logprob is 10-based log of $P(f \mid c)$. 
        
  \item sys\_output is the classification result on the training and
        test data (cf. {\bf sys1}). Each line has the following format:\\
	{\tt instanceName true\_class\_label c1 p1 c2 p2 ...,} \\
        where $p_i=P(c_i \mid x)=\frac{P(c_i, x)}{P(x)}$.
        The $(c_i, p_i)$ pairs should be sorted according to 
        the value of $p_i$ in descending order. 

  \item acc\_file shows the confusion matrix and the accuracy for
       the training and the test data (cf. {\bf acc1}). 

  \item As always, {\bf model1}, {\bf sys1}, and {\bf acc1}
        are NOT gold standard.
        These files were created with a much smaller training dataset.
\end{itemize}


 Run build\_NB1.sh with {\bf train.vectors.txt} as the 
training data, {\bf test.vectors.txt} as the test data,
and class\_prior\_delta set to 0:
\begin{itemize}
\item Fill out Table 1 with different values of cond\_prob\_delta.
\item Store the model\_file, sys\_output
and acc\_file for the second row (when cond\_prob\_delta is 0.5) under
q2/.
\end{itemize}


\begin{table}[h]
\centering
\caption{Results of your {\bf Bernoulli} NB model}
\label{table1}
\begin{tabular}{|r|l|l|} \hline
cond\_prob\_delta   & Training accuracy & Test accuracy \\ \hline
0.1       &  &    \\ \hline
0.5       &  &    \\ \hline
1.0      &  &    \\ \hline
%% 2.0     &  &    \\ \hline
\end{tabular}
\end{table}


%%%%%%%%%%%%%%%%%%%%%%%%
\vspace{0.4 in}

\noindent {\bf Q3 (35 points):} Write a script, {\bf build\_NB2.sh}, 
  that implements the multinomial NB model. Other than the modeling
  (e.g., the features in the multinomial NB model are real-valued),
  everything else (e.g., the input/output files) is the same as in Q2.
  \begin{itemize}
  \item  Fill out Table 2.
  \item Store the model\_file, sys\_output
and acc\_file for the second row (when cond\_prob\_delta is 0.5) under
q3/.
  \end{itemize}
  

\begin{table}[h]
\centering
\caption{Results of your {\bf multinomial} NB model}
\label{table1}
\begin{tabular}{|r|l|l|} \hline
cond\_prob\_delta   & Training accuracy & Test accuracy \\ \hline
0.1       &  &    \\ \hline
0.5       &  &    \\ \hline
1.0     &  &    \\ \hline
%% 2.0     &  &    \\ \hline
\end{tabular}
\end{table}


%%%%%%%%%%%%%%%%%%%%%%%%
\vspace{0.5 in}
\noindent {\bf Submission:} Submit the following to Canvas:

\begin{itemize}
  \item Your note file {\it readme.(txt $\mid$ pdf)}
      that includes Table 1 and 2, 
      and any notes that you want the TA to read.
      

  \item  hw3.tar.gz that includes all the files specified in
      dropbox/19-20/572/hw3/submit-file-list, plus any source code
      (and binary code) used by the shell scripts.

  \item Make sure that you run {\bf check\_hw3.sh} before
       submitting your hw.tar.gz.
        
    
\end{itemize}

\end{document}




\documentclass[11pt]{article}
\usepackage{hyperref}
\usepackage[margin=0.75in]{geometry}
\usepackage{amsmath}

\begin{document}

\title{LING 575K HW6}
\date{\vspace{-0.2in}Due 11PM on May 13, 2021}
\maketitle


\noindent In this assignment, you will 
\begin{itemize}
  \item Develop understanding of recurrent neural networks
  \item Implement components of data processing 
  \item Implement key pieces of two variants of a recurrent model architecture
\end{itemize}
All files referenced herein may be found in \texttt{/dropbox/20-21/575k/hw6/} on patas.


\section{Recurrent Neural Network Encoders}

\noindent {\bf Q1: }  


\section{Implementing an RNN Sentiment Classifier}

blah blah blah.

\vspace{2em}
\noindent {\bf Q1: Data processing} 



\section{Running the Classifier}

\texttt{run.py} contains a basic training loop for a feed-forward language model, which will record the training loss and generate text every $N$ epochs (controlled by the flat \texttt{--generate\_every}, set to 4 by default).

\vspace{2em}
\noindent {\bf Q1: Vanilla RNN}  

\vspace{2em}
\noindent {\bf Q2: LSTM}  

\vspace{2em}
\noindent {\bf Q3: Modify one hyper-parameter} Re-run the training loop, modifying one of the following hyper-parameters, which are specified by command-line flags:
\begin{itemize}
  \item Hidden layer size
  \item Embedding size
  \item Number of epochs
\end{itemize}


\section{Testing your code}

In the dropbox folder for this assignment, we will include a file \texttt{test\_all.py} with a few very simple unit tests for the methods that you need to implement.  You can verify that your code passes the tests by running \texttt{pytest} from your code's directory, with the course's conda environment activated.


%%%%%%%%%%%%%%%%%%%%%%%%%%%%%%%
\section*{Submission Instructions}

In your submission, include the following:
\begin{itemize}
  \item readme.(txt$\mid$pdf) that includes your answers to \S1 and \S3. 
  \item \texttt{hw5.tar.gz} containing:
  \begin{itemize}
    \item run\_hw5.sh.  This should contain the code for activating the conda environment and your run commands for \S3 above.  You can use run\_hw2.sh from the previous assignment as a template.
    \item data.py
    \item model.py
    \item ops.py
    \item run.py
  \end{itemize}
\end{itemize}





\end{document}




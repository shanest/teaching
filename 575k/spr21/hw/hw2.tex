\documentclass[11pt]{article}
\usepackage{hyperref}
\usepackage[margin=0.75in]{geometry}

\begin{document}

\title{LING 575K HW2}
\date{\vspace{-0.2in}Due 11PM on Apr 15, 2021}
\maketitle


In this assignment, you will answer some written questions about and then implement word2vec; in particular, the method \emph{skip-gram with negative sampling (SGNS)}.  By doing so you will:
\begin{itemize}
  \item Count parameters
  \item Take derivatives of a loss
  \item Translate mathematics into implemented code
  \item Train your own set of word vectors and briefly analyze them
\end{itemize}

\section{Understanding Word2Vec}

\noindent {\bf Q1: Parameters}  How many parameters are there in the SGNS model?  Write your answer in terms of $V$ (the vocabulary) and $d_e$, the embedding dimension.  [Hint: one parameter is \emph{a single real number}.]


\vspace{2em}
\noindent {\bf Q2: Sigmoid}  Sigmoid is the logistic curve $\sigma(x) = \frac{1}{1+e^{-x}}$.
\begin{itemize}
  \item What is the range of $\sigma(x)$?
  \item How is it used in the SGNS model?
  \item Compute $\frac{d\sigma}{dx}$; show your work.  [Hint: write your final answer in terms of $\sigma(x)$.]
\end{itemize}

\vspace{2em}
\noindent {\bf Q3: Loss function}


\section{Implementing Word2Vec}

\vspace{2em}
\noindent {\bf Q1: } 

%%%%%%%%%%%%%%%%%%%%%%%%%%%%%%%
\section*{Submission Instructions}

In your submission, include the following:
\begin{itemize}
  \item readme.(txt$\mid$pdf) that includes your answers to \S1. 
  \item \texttt{hw2.tar.gz} containing:
  \begin{itemize}
    \item 
  \end{itemize}
\end{itemize}





\end{document}




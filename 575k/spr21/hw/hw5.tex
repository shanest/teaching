\documentclass[11pt]{article}
\usepackage{hyperref}
\usepackage[margin=0.75in]{geometry}
\usepackage{amsmath}

\begin{document}

\title{LING 575K HW5}
\date{\vspace{-0.2in}Due 11PM on May 6, 2021}
\maketitle


\noindent In this assignment, you will 
\begin{itemize}
  \item Develop understanding of a feed-forward neural language model
  \item Implement components of data processing and text generation
  \item Implement key pieces of the model architecture
\end{itemize}
All files referenced herein may be found in \texttt{/dropbox/20-21/575k/hw5/} on patas.


\section{Understanding the Feed-Forward Language Model}

\noindent {\bf Q1: }  


\vspace{2em}
\noindent {\bf Q2: tanh} The model uses the hyperbolic tangent (tanh) activation function, defined as:
\[ \tanh(x) = \frac{e^a - e^{-a}}{e^a + e^{-a}} \]
\begin{itemize}
  \item Show that $\tanh(x) = 2\sigma(2x) - 1$, where $\sigma(x)$ is the sigmoid function.
  \item Show that $\frac{d}{dx}\tanh(x) = 1 - \tanh^2(x)$.
\end{itemize}


\section{Implementing the Feed-Forward Language Model}

\noindent {\bf Q1: Data processing} The basic ingredient of a language model is a dataset of next-token predictions. In \texttt{data.py}, you will find a basic dataset class SSTLanguageModelingDataset.  In its from\_file method, it iterates through the lines in a file, and calls a helper function to generate example pairs.
\begin{itemize}
  \item Implement the method \texttt{examples\_from\_characters}.  Read the docstring closely for desired behavior.
\end{itemize}

\vspace{2em}
\noindent {\bf Q2: Implementing tanh}  In \texttt{ops.py}, you will find a skeleton Operation for tanh.  Using your written answer above as a guide, implement the forward and backward methods for this op.

\vspace{2em}
\noindent {\bf Q3: Implementing the Language Model} In \texttt{model.py}, you will find the main model class FeedForwardLanguageModel, with its initialization method written.  Implement the \texttt{.forward} method, using its docstring as a guide.  [Hint: \texttt{ops.concat}, which we provide, will be necessary.]

\vspace{2em}
\noindent {\bf Q4: Generating the next character}


\section{Running the Language Model}



\section{Testing your code}

In the dropbox folder for this assignment, we will include a file \texttt{test\_all.py} with a few very simple unit tests for the methods that you need to implement.  You can verify that your code passes the tests by running \texttt{pytest} from your code's directory, with the course's conda environment activated.


%%%%%%%%%%%%%%%%%%%%%%%%%%%%%%%
\section*{Submission Instructions}

In your submission, include the following:
\begin{itemize}
  \item readme.(txt$\mid$pdf) that includes your answers to . 
  \item \texttt{hw5.tar.gz} containing:
  \begin{itemize}
    \item run\_hw5.sh.  This should contain the code for activating the conda environment and your run commands for XX above.  You can use run\_hw2.sh from the previous assignment as a template.
    \item data.py
    \item model.py
    \item ops.py
    \item run.py
  \end{itemize}
\end{itemize}





\end{document}




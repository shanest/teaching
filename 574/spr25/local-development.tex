\documentclass[11pt]{article}
\usepackage{hyperref}
\usepackage[margin=0.75in]{geometry}

\begin{document}

\title{LING 574 Deep Learning for NLP}
\date{\vspace{-0.2in}Local Development Instructions}
\author{}
\maketitle


\paragraph{Nota bene:} These instructions have only been tested on macOS and Linux.  In theory, they should also work on Windows with miniconda, though it may be helpful to use the Windows Subystem for Linux (WSL) in that case.

\section{Download Miniconda}

Install Anaconda. This is a necessary component for running the code for most assignments in this course.  We have setup a conda environment for the assignment, but you will need to install anaconda in order to use that environment.  These are ``free points''.  From your home directory, please execute the following steps:
\begin{enumerate}
	\item \texttt{wget https://repo.anaconda.com/miniconda/Miniconda3-latest-Linux-x86\_64.sh}
	\item \texttt{sh Miniconda3-latest-Linux-x86\_64.sh}
\end{enumerate}


\section{Setting up the Environment}

Download `environment.yml' either from the course dropbox folder or from Canvas.  Then, from the directory where you downloaded `environment.yml', execute the following command: \texttt{conda env create -f environment.yml}.  This will create a new conda environment called `574-env'.  To activate the environment, execute \texttt{conda activate 574-env}.  You can deactivate the environment with \texttt{conda deactivate}.


\section{Testing on Patas}

While this should provide you the same environment that we have for the course on patas, you should still confirm that your homework runs on patas, since that is where we will be grading.

\section{Developing on Patas}

You can also develop directly on patas, using either SSH plus a terminal-based editor, or e.g.\ VSCode and its SSH extension.


\end{document}



